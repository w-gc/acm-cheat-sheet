% \documentclass[8pt,adobefonts,fancyhdr,hyperref,UTF8]{ctexbook}
\documentclass[10pt,fancyhdr,hyperref,UTF8]{ctexbook}

\usepackage{multirow}
% for \soul 删除线
\usepackage{ulem}
% 表头斜线
\usepackage{diagbox}

\makeatletter
\input{format.cls}
\makeatother

%--------------------------------------------------------------
\usepackage{color}
\newcommand{\Note}[1]{\textcolor{blue}{【注:#1】}} %% 笔记
\newcommand{\Hl}[1]{\textcolor{red}{#1}} %% Highlight
\newcommand{\Dl}{---------------------------------------------------------------------分割线-------------------------------------------------------------------} %% Dividingline
\usepackage{listings}
\lstnewenvironment{NoteCode}{
\lstset{ %
backgroundcolor=\color{white},   % choose the background color
basicstyle=\small,        % size of fonts used for the code
columns=fullflexible,
breaklines=true,                 % automatic line breaking only at whitespace
captionpos=b,                    % sets the caption-position to bottom
tabsize=4,
commentstyle=\color{magenta},    % comment style
escapeinside={\%*}{*)},          % if you want to add LaTeX within your code
keywordstyle=\color{blue},       % keyword style
stringstyle=\color{green}\ttfamily,     % string literal style
frame=single, % none
% rulesepcolor=\color{red!20!green!20!blue!20},
% identifierstyle=\color{red},
language=C++
}}{}

%---------------------------上面的现实注释,下面的不现实注释---------------

% \newcommand{\Note}[1]{} %% 笔记 (注释掉)
% \newcommand{\Hl}[1]{#1} %% Highlight (注释掉)
% \newcommand{\Dl}[1]{} %% Dividingline (注释掉)
% \usepackage{comment}
% \newenvironment{NoteCode}{}{}
% \excludecomment{NoteCode}
%-------------------------------------------------------------



\begin{document}
\sloppy
\newcommand\BookTitle{手写代码必备手册}
\pagestyle{fancy}
\fancyhf{}
\fancyhead[RE]{\normalfont\small\rmfamily\nouppercase{\leftmark}}
\fancyhead[LO]{\normalfont\small\rmfamily\nouppercase{\rightmark}}
\fancyhead[LE,RO]{\thepage}
%\fancyfoot[LE,LO]{\small\normalfont\youyuan\BookTitle}
%\fancyfoot[RE,RO]{\textsf{\small \color{blue} https://github.com/soulmachine/acm-cheatsheet}}

\makeatletter
\@openrightfalse
\makeatother

\frontmatter % 开始前言目录,页码用罗马数字

% \include{title}

\tableofcontents

\mainmatter % 开始正文,页码用阿拉伯数字

\graphicspath{{diagrams/}}

% \include{chapProgrammingTrick}
% \chapter{线性表}
线性表(Linear List)包含以下几种:
\begindot
\item 顺序存储:数组
\item 链式存储:单链表,双向链表,循环单链表,循环双向链表
\item 二者结合:静态链表
\myenddot

\Note{
std::array Defined in header <array>
}
\begin{NoteCode}
template <class Ty, std::size_t N>
class array;
\end{NoteCode}

\Note{\\
\textbf{parameters}: \\
\fn{Ty} 元素的类型; \\
\fn{N} 元素数量。\\
\\
\textbf{Members}: \\
\underline{const_iterator}	The type of a constant iterator for the controlled sequence.( \fn{for (Array::const_iterator it = a.begin(); it != a.end(); ++it);} )\\
\underline{const_pointer}	The type of a constant pointer to an element. ( \fn{Array::const_pointer ptr = \&*a.begin();} ) \\
\underline{const_reference}	The type of a constant reference to an element. ( \fn{Array::const_reference ref = *a.begin();} )\\
\underline{const_reverse_iterator}	The type of a constant reverse iterator for the controlled sequence. ( \fn{Array::const_reverse_iterator it = a.rbegin();} ) \\
\underline{difference_type}	The type of a signed distance between two elements. ( \fn{Array::difference_type diff = a.begin() - a.end();} )\\
\underline{iterator}	The type of an iterator for the controlled sequence. ( \fn{for (Array::iterator it = a.begin(); it != a.end(); ++it)} ) \\
\underline{pointer}	The type of a pointer to an element. ( \fn{Array::pointer ptr = \&*a.begin();} ) \\
\underline{reference}	The type of a reference to an element. (\fn{Array::reference ref = *a.begin();} ) \\
\underline{reverse_iterator}	The type of a reverse iterator for the controlled sequence. ( \fn{Array::reverse_iterator it = a.rbegin();} ) \\
\underline{size_type}	The type of an unsigned distance between two elements. ( \fn{Array::size_type diff = a.end() - a.begin();} ) \\
\underline{value_type}	The type of an element. (\fn{Array::value_type val = *a.begin();}) \\
\\
\textbf{Member Function}: \\
\underline{array}	Constructs an array object. \\
\underline{assign}	(Obsolete. Use fill.) Replaces all elements. \\
\underline{at}	Accesses an element at a specified position. ( \fn{a.at(1);} ) \\
\underline{back}	Accesses the last element. \\
\underline{begin}	Designates the beginning of the controlled sequence. \\
\underline{cbegin}	Returns a random-access const iterator to the first element in the array. \\
\underline{cend}	Returns a random-access const iterator that points just beyond the end of the array. \\
\underline{crbegin}	Returns a const iterator to the first element in a reversed array. \\
\underline{crend}	Returns a const iterator to the end of a reversed array. \\
\underline{data}	Gets the address of the first element. \\
\underline{empty}	Tests whether elements are present. \\
\underline{end}	Designates the end of the controlled sequence. \\
\underline{fill}	Replaces all elements with a specified value. \\
\underline{front}	Accesses the first element. \\
\underline{max_size}	Counts the number of elements. \\
\underline{rbegin}	Designates the beginning of the reversed controlled sequence. \\
\underline{rend}	Designates the end of the reversed controlled sequence. \\
\underline{size}	Counts the number of elements. \\
\underline{swap}	Swaps the contents of two containers. \\
\\
\textbf{Operator}:\\
\underline{array::operator=}	Replaces the controlled sequence.\\
\underline{array::operator[]}	Accesses an element at a specified position.\\
}

\Dl

\Note{
std::vector Defined in header <vector>
}

\begin{NoteCode}
template <class Type, class Allocator = allocator<Type>>
class vector
\end{NoteCode}

\Note{
\textbf{Member functions}\\
\underline{assign}	Erases a vector and copies the specified elements to the empty vector.\\
\underline{at}	Returns a reference to the element at a specified location in the vector.\\
\underline{back}	Returns a reference to the last element of the vector.\\
\underline{begin}	Returns a random-access iterator to the first element in the vector.\\
\underline{capacity}	Returns the number of elements that the vector could contain without allocating more storage.\\
\underline{cbegin}	Returns a random-access const iterator to the first element in the vector.\\
\underline{cend}	Returns a random-access const iterator that points just beyond the end of the vector.\\
\underline{crbegin}	Returns a const iterator to the first element in a reversed vector.\\
\underline{crend}	Returns a const iterator to the end of a reversed vector.\\
\underline{clear}	Erases the elements of the vector.\\
\underline{data}	Returns a pointer to the first element in the vector.\\
\underline{emplace}	Inserts an element constructed in place into the vector at a specified position. ( \fn{vv1.emplace( vv1.begin(), move( vector <int>(1,2) ) );} ) \\
\underline{emplace_back}	Adds an element constructed in place to the end of the vector. ( \fn{v.emplace_back(1, 3.14);} ) \\
\underline{empty}	Tests if the vector container is empty.\\
\underline{end}	Returns a random-access iterator that points to the end of the vector.\\
\underline{erase}	Removes an element or a range of elements in a vector from specified positions. ( \fn{iterator erase(const_iterator position); \\
iterator erase(const_iterator first,const_iterator last);} ) \\
\underline{front}	Returns a reference to the first element in a vector.\\
\underline{get_allocator}	Returns an object to the allocator class used by a vector.\\
\underline{insert}	Inserts an element or many elements into the vector at a specified position.( \fn{\\iterator insert(const_iterator position, const Type\& value);\\
iterator insert(const_iterator position, Type\&\& value);\\
void insert(const_iterator position, size_type count, const Type\& value);\\
template <class InputIterator>
void insert(const_iterator position, InputIterator first,InputIterator last);
} )\\
\underline{max_size}	Returns the maximum length of the vector.\\
\underline{pop_back}	Deletes the element at the end of the vector.\\
\underline{push_back}	Add an element to the end of the vector.\\
\underline{rbegin}	Returns an iterator to the first element in a reversed vector.\\
\underline{rend}	Returns an iterator to the end of a reversed vector.\\
\underline{reserve}	Reserves a minimum length of storage for a vector object.\\
\underline{resize}	Specifies a new size for a vector.\\
\underline{shrink_to_fit}	Discards excess capacity.(丢弃空闲内存)\\
\underline{size} Returns the number of elements in the vector.\\
\underline{swap}	Exchanges the elements of two vectors.\\
}


\Dl

\Note{std::deque Defined in header <deque>}

\begin{NoteCode}
template <class Type, class Allocator =allocator<Type>>
class deque
\end{NoteCode}

\Note{
\\
\textbf{Member functions}\\
$\cdots$\\
\underline{emplace}	Inserts an element constructed in place into the deque at a specified position.\\
\underline{emplace_back}	Adds an element constructed in place to the end of the deque.\\
\underline{emplace_front}	Adds an element constructed in place to the start of the deque.\\
\underline{insert}	Inserts an element, several elements, or a range of elements into the deque at a specified position.\\
\underline{max_size}	Returns the maximum possible length of the deque.\\
\underline{pop_back}	Erases the element at the end of the deque.\\
\underline{pop_front}	Erases the element at the start of the deque.\\
\underline{push_back}	Adds an element to the end of the deque.\\
\underline{push_front}	Adds an element to the start of the deque.\\
\underline{resize}	Specifies a new size for a deque.\\
$\cdots$\\
}

\Dl

\Note{std::list Defined in header <list>}

\begin{NoteCode}
template <class Type, class Allocator= allocator<Type>>
class list
\end{NoteCode}

\Note{
\\
\textbf{Member functions}\\
$\cdots$\\
\underline{merge}	Removes the elements from the argument list, inserts them into the target list, and orders the new, combined set of elements in ascending order or in some other specified order. ( \fn{\\
void merge(list<Type, Allocator>\& right);\\
template <class Traits>\\
void merge(list<Type, Allocator>\& right, Traits comp);
} ) \\
\underline{remove}	Erases elements in a list that match a specified value. (\fn{void remove(const Type\& val);}) \\
\underline{remove_if}	Erases elements from the list for which a specified predicate is satisfied. (\fn{template <class Predicate>
void remove_if(Predicate pred);\\
l.remove_if( is_odd<int>( ) );}) \\
\underline{reverse}	Reverses the order in which the elements occur in a list.\\
\underline{sort}	Arranges the elements of a list in ascending order or with respect to some other order relation. ( \fn{void sort(); \\
template <class Traits>
    void sort(Traits comp);} )\\
\underline{splice}	Removes elements from the argument list and inserts them into the target list. (\fn{ \\
// insert the entire source list 会在position后把list\&x所有的元素到剪接到要操作的list对象\\
void splice(const_iterator Where, list<Type, Allocator>\& Source);\\
void splice(const_iterator Where, list<Type, Allocator>\&\& Source);\\
// insert one element of the source list 只会把it的值剪接到要操作的list对象中\\
void splice(const_iterator Where, list<Type, Allocator>\& Source, const_iterator Iter);\\
void splice(const_iterator Where, list<Type, Allocator>\&\& Source, const_iterator Iter);\\
// insert a range of elements from the source list 把first 到 last 剪接到要操作的list对象中\\
void splice(const_iterator Where, list<Type, Allocator>\& Source, const_iterator First, const_iterator Last);\\
void splice(const_iterator Where, list<Type, Allocator>\&\& Source, const_iterator First, const_iterator Last);
})\\
\underline{unique}	Removes adjacent duplicate elements or adjacent elements that satisfy some other binary predicate from the list.(连续重复\fn{void unique(); \\
template <class BinaryPredicate>
void unique(BinaryPredicate pred);})\\
$\cdots$\\
}


\Dl

\Note{std::forward_list Defined in header <forward_list>}

\begin{NoteCode}
template <class Type, class Allocator = allocator<Type>>
class forward_list
\end{NoteCode}

\Note{
\\
\textbf{Member functions}\\
$\cdots$\\
\underline{before_begin}	Returns an iterator addressing the position before the first element in a forward list.\\
\underline{emplace_after}	Move constructs a new element after a specified position.\\
\underline{emplace_front}	Adds an element constructed in place to the beginning of the list.\\
\underline{erase_after}	Removes elements from the forward list after a specified position.\\
\underline{insert_after}	Adds elements to the forward list after a specified position.\\
\underline{merge}	Removes the elements from the argument list, inserts them into the target forward list, and orders the new, combined set of elements in ascending order or in some other specified order.\\
\underline{pop_front}	Deletes the element at the beginning of a forward list.\\
\underline{push_front}	Adds an element to the beginning of a forward list.\\
\underline{splice_after}	Restitches links between nodes.\\
$\cdots$\\
}
\chapter{字符串}

\section{ACM题输入处理}
\Note{此章节均为补充内容。主要来源于\myurl{https://blog.csdn.net/jeffscott/article/details/107969136}}

\Note{删除字符串}
\begin{NoteCode}
string str1="God bless you!";
string str2="Father be with you!";
str1.erase(3);//删除[3]及以后的字符,并返回新的字符串
str2.erase(3,5);//删除从 [3] 开始的后 5 个字符,并返回新字符串
// God
// Fate with you!
\end{NoteCode}

\Note{追加字符(串)}
\begin{NoteCode}
string str = "God bless you!";
str.push_back('&');//在str的末尾添加字符'&',注意只能是字符,不能是字符串如"&"
str.append("Father be with you!");//在 str 末尾添加字符串
// God bless you!&
// God bless you!&Father be with you!
\end{NoteCode}

\Note{插入字符串}
\begin{NoteCode}
string str = "God bless you!";
str.insert(0,"Father&");//从位置0开始添加字符串
str.insert(0,"Maybe",3);//从位置0开始插入字符串"Maybe”的前三个即“May”
str.insert(0,"You Wish",4,4);//从位置0开始插入子串"You Wish"从[4]开始的后四个字符,即"Wish"
// Father&God bless you!
// MayFather&God bless you!
// WishMayFather&God bless you!
\end{NoteCode}

\Note{替换字符串}
\begin{NoteCode}
string str = "God bless you!";
str.replace(0,3,"Father");//把str从[0]开始的3个字符,即God替换为Father
str.replace(0,6,"God be with you!",3);//把str从[0]开始的6个字符,即Father替换God
// Father bless you!
// God bless you!
\end{NoteCode}


\Note{获取字符串子串}
\begin{NoteCode}
string str = "God bless you!";
cout<<str.substr(4)<<endl;//返回从[4]开始的后所有字符,即bless you!
cout<<str.substr(4,5)<<endl;//返回[4]以后的5个字符,即bless
\end{NoteCode}

\Note{查找子串}
\begin{NoteCode}
string str = "God bless you!";
cout<<str.find("God")<<endl; // 返回God在str中的起始位置,即0
cout<<str.find("God",3)<<endl; // 在 str[3]~str[n-1] 范围内查找并返回字符串 God 在 str 的位置,即npos=18446744073709551615
cout<<string::npos<<endl; // 18446744073709551615
cout<<str.rfind("God",3)<<endl; // 在 str[0]~str[3] 范围内查找并返回字符串 God 在 str 的位置,即0
\end{NoteCode}

\Note{去除一个字符串中的所有空格}
\begin{NoteCode}
/*返回去除空格的字符串*/
string rmSpace(string s){
    while(s.find(" ")!=s.npos)
        s.replace(s.find(" "),1,"");
    return s;
}
\end{NoteCode}

\Note{将一个字符串统一大小写}
\begin{NoteCode}
string str;
getline(cin,str);
transform(str.begin(),str.end(),str.begin(),::toupper);//统一转换为大写
transform(str.begin(),str.end(),str.begin(),::tolower);//统一转换为小写
\end{NoteCode}

\Note{字符串分割}
\begin{NoteCode}
#include<iostream>
#include<string>
#include<vector>
 
vector<string> split(const string &str, conststring &pattern){
    vector<string> res;
    if(str == "") return res;
    //在字符串末尾也加入分隔符,方便截取最后一段
    string strs = str + pattern;
    size_t pos = strs.find(pattern);
    while(pos != strs.npos){
        string temp = strs.substr(0, pos);
        res.push_back(temp);
        //去掉已分割的字符串,在剩下的字符串中进行分割
        strs = strs.substr(pos+1, strs.size());
        pos = strs.find(pattern);
    }
    return res;
}
\end{NoteCode}

\Note{字符串类型转换}
\begin{NoteCode}
#include<iostream>
#include<string>
using namespace std;

int main(){
    //int --> string
    int i = 5;
    string s = to_string(i);
    cout << s << endl;
    //double --> string
    double d = 3.14;
    cout << to_string(d) << endl;
    //long --> string
    long l = 123234567;
    cout << to_string(l) << endl;
    //char --> string
    char c = 'a';
    cout << to_string(c) << endl;   //自动转换成int类型的参数
    //char --> string
    string cStr; cStr += c;
    cout << cStr << endl;
 
    s = "123.257";
    //string --> int;
    cout << stoi(s) << endl;
    //string --> long
    cout << stol(s) << endl;
    //string --> float
    cout << stof(s) << endl;
    //string --> doubel
    cout << stod(s) << endl;
 
  return 0;
}
\end{NoteCode}

\Note{string转char*}
\begin{NoteCode}
string str;
char *c = str.c_str();
\end{NoteCode}

\Note{判断字符的类型}
\begin{NoteCode}
cout << isdigit('8') << endl; // 判断是否是数字
cout << isalnum('g') << endl; // 判断是否是字母或数字
cout << isspace(' ') << endl; // 判断是否是空格
cout << islower('a') << endl; // 判断是否是小写
cout << isupper('A') << endl; // 判断是否是大写
cout << isalpha('y') << endl; // 判断是否是字母
cout << tolower('G') << endl; // 将大写字母转为小写字母
\end{NoteCode}

\Note{[1,2,3,4]}
\begin{NoteCode}
#include<sstream>
// ... 
int main(){
    string str;
    getline(cin, str);
    str = str.substr(1, str.size() - 2) + ",";
    string tmp;
    stringstream ss(str);

    while( getline(ss, tmp, ',') ) cout << tmp;
    return 0;
}
// [1,2,3,4]
// 1234 
\end{NoteCode}




\section{字符串API} %%%%%%%%%%%%%%%%%%%%%%%%%%%%%%
面试中经常会出现,现场编写 \fn{strcpy, strlen, strstr, atoi} 等库函数的题目。这类题目看起来简单,实则难度很大,区分都很高,很容易考察出你的编程功底,是面试官的最爱。


\subsection{strlen}


\subsubsection{描述}
实现 \fn{strlen},获取字符串长度,函数原型如下:
\begin{Code}
size_t strlen(const char *str);
\end{Code}


\subsubsection{分析}



\subsubsection{代码}
\begin{Code}
size_t strlen(const char *str) {
    const char *s;
    for (s = str; *s; ++s) {}
    return(s - str);
}
\end{Code}

\Note{
1. \fn{const int*p=\&a}(在*前面);当把 \fn{const} 放最前面的时候,它修饰的就是 \fn{*p},\fn{*p} 表示的是指针变量 \fn{p} 所指向的内存单元里面的内容,此时这个内容不可变。\\
2. \fn{int*const p=\&a};此时 \fn{const} 修饰的是 \fn{p},所以 \fn{p} 中存放的内存单元的地址不可变,而内存单元中的内容可变。(变成一个引用)\\
3. \fn{const int*const p=\&a};此时 \fn{*p} 和 \fn{p} 都被修饰了,那么 \fn{p} 中存放的内存单元的地址和内存单元中的内容都不可变。
}

\subsection{strcpy}


\subsubsection{描述}
实现 \fn{strcpy},字符串拷贝函数,函数原型如下:
\begin{Code}
char* strcpy(char *to, const char *from);
\end{Code}


\subsubsection{分析}



\subsubsection{代码}
\begin{Code}
char* strcpy(char *to, const char *from) {
    assert(to != NULL && from != NULL);
    char *p = to;
    while ((*p++ = *from++) != '\0');
    return to;
}
\end{Code}


\subsection{strstr}


\subsubsection{描述}
实现 \fn{strstr},子串查找函数\Note{在\fn{haystack}中找\fn{needle}},函数原型如下:
\begin{Code}
char * strstr(const char *haystack, const char *needle);
\end{Code}

\subsubsection{分析}
暴力算法的复杂度是 $O(m*n)$,代码如下。其他算法见第\S \ref{sec:substring-search}节 “子串查找”。

\subsubsection{代码}
\begin{Code}
char *strstr(const char *haystack, const char *needle) {
    // if needle is empty return the full string
    // if (!*needle) return (char*) haystack;

    const char *p1;
    const char *p2;
    const char *p1_advance = haystack;
    for (p2 = &needle[1]; *p2; ++p2) {
        p1_advance++;   // advance p1_advance M-1(strlen(needle)-1) times
    }

    for (p1 = haystack; *p1_advance; p1_advance++) {
        char *p1_old = (char*) p1;
        p2 = needle;
        while (*p1 && *p2 && *p1 == *p2) {
            p1++;
            p2++;
        }
        if (!*p2) return p1_old;

        p1 = p1_old + 1;
    }
    return NULL;
}
\end{Code}
\begin{NoteCode}
int strStr(string haystack, string needle) {
    if(needle.empty()) return 0;
    if(haystack.empty() || haystack.length() < needle.length()) return -1;
    const char *p1;
    const char *p2;
    const char *p1_advance = &haystack[needle.length() - 1];
    for (p1 = &haystack[0]; *p1_advance; p1_advance++) {
        char *p1_old = (char*) p1;
        p2 = &needle[0];
        while (*p1 && *p2 && *p1 == *p2) {
            p1++;
            p2++;
        }
        if (!*p2) return p1_old - &haystack[0];
        p1 = p1_old + 1;
    }
    return -1;
}
\end{NoteCode}


\subsubsection{相关题目}
与本题相同的题目:
\begindot
\item LeetCode Implement strStr(), \myurl{http://leetcode.com/oldoj\#question_28}
\myenddot

与本题相似的题目:
\begindot
\item  无
\myenddot


\subsection{atoi}
\label{sec:string-to-integer}


\subsubsection{描述}
实现 \fn{atoi},将一个字符串转化为整数,函数原型如下:
\begin{Code}
int atoi(const char *str);
\end{Code}


\subsubsection{分析}
注意,这题是故意给很少的信息,让你来考虑所有可能的输入。

来看一下\fn{atoi}的官方文档(\myurl{http://www.cplusplus.com/reference/cstdlib/atoi/}),看看它有什么特性:

The function first discards as many whitespace characters as necessary until the first non-whitespace character is found. Then, starting from this character, takes an optional initial plus or minus sign followed by as many numerical digits as possible, and interprets them as a numerical value.

The string can contain additional characters after those that form the integral number, which are ignored and have no effect on the behavior of this function.

If the first sequence of non-whitespace characters in str is not a valid integral number, or if no such sequence exists because either str is empty or it contains only whitespace characters, no conversion is performed.

If no valid conversion could be performed, a zero value is returned. If the correct value is out of the range of representable values, \fn{INT_MAX (2147483647)} \Note{\#7FFF FFFF} or \fn{INT_MIN (-2147483648)} \Note{\# 8000 0000} is returned.

注意几个测试用例:
\begin{enumerate}
\item 不规则输入,但是有效,"-3924x8fc", "  +  413",
\item 无效格式," ++c", " ++1"
\item 溢出数据,"2147483648"
\end{enumerate}


\subsubsection{代码}
\begin{Code}
int atoi(const char *str) {
    int num = 0;
    int sign = 1;
    const int len = strlen(str);
    int i = 0;

    while (str[i] == ' ' && i < len) i++; //删除前面的空格

    if (str[i] == '+') i++;

    if (str[i] == '-') {
        sign = -1;
        i++;
    }

    for (; i < len; i++) {
        if (str[i] < '0' || str[i] > '9') break;
        if (num > INT_MAX / 10 ||                     //再增加一位便溢出
                (num == INT_MAX / 10 &&
                        (str[i] - '0') > INT_MAX % 10)) {
            return sign == -1 ? INT_MIN : INT_MAX;
        }
        num = num * 10 + str[i] - '0';
    }
    return num * sign;
}
\end{Code}


\subsubsection{相关题目}
与本题相同的题目:
\begindot
\item LeetCode String to Integer (atoi), \myurl{http://leetcode.com/oldoj\#question_8}
\myenddot

与本题相似的题目:
\begindot
\item  无
\myenddot


\section{字符串排序} %%%%%%%%%%%%%%%%%%%%%%%%%%%%%%


\section{单词查找树} %%%%%%%%%%%%%%%%%%%%%%%%%%%%%%


\section{子串查找} %%%%%%%%%%%%%%%%%%%%%%%%%%%%%%
\label{sec:substring-search}

字符串的一种基本操作就是\textbf{子串查找}(substring search):给定一个长度为$N$的文本和一个长度为$M$的模式串(pattern string),在文本中找到一个与该模式相符的子字符串。

最简单的算法是暴力查找,时间复杂度是$O(MN)$。下面介绍两个更高效的算法。


\subsection{KMP算法}
KMP算法是Knuth、Morris和Pratt在1976年发表的。它的基本思想是,当出现不匹配时,就能知晓一部分文本的内容(因为在匹配失败之前它们已经和模式相匹配)。我们可以利用这些信息避免将指针回退到所有这些已知的字符之前。这样,当出现不匹配时,可以提前判断如何重新开始查找,而这种判断只取决于模式本身。

详细解释请参考《算法》\footnote{《算法》,Robert Sedgewick,人民邮电出版社,\myurl{http://book.douban.com/subject/10432347/}}第5.3.3节。这本书讲的是确定有限状态自动机(DFA)的方法。

推荐网上的几篇比较好的博客,讲的是部分匹配表(partial match table)的方法(即next数组),“字符串匹配的KMP算法” \myurl{http://t.cn/zTOPfdh},图文并茂,非常通俗易懂,作者是阮一峰;“KMP算法详解” \myurl{http://www.matrix67.com/blog/archives/115},作者是顾森 Matrix67;"Knuth-Morris-Pratt string matching" \myurl{http://www.ics.uci.edu/~eppstein/161/960227.html}。

使用next数组的KMP算法的C语言实现如下。
\begin{Codex}[label=kmp.c]
#include <stdio.h>
#include <stdlib.h>
#include <string.h>

/*
 * @brief 计算部分匹配表,即next数组.
 *
 * @param[in] pattern 模式串
 * @param[out] next next数组
 * @return 无
 */
void compute_prefix(const char *pattern, int next[]) {
    int i;
    int j = -1;
    const int m = strlen(pattern);

    next[0] = j;
    for (i = 1; i < m; i++) {
        // 不匹配时回溯
        while (j > -1 && pattern[j + 1] != pattern[i]) j = next[j];
        // 匹配时j++
        if (pattern[i] == pattern[j + 1]) j++;
        next[i] = j;
    }
}

/*
 * @brief KMP算法.
 *
 * @param[in] text 文本
 * @param[in] pattern 模式串
 * @return 成功则返回第一次匹配的位置,失败则返回-1
 */
int kmp(const char *text, const char *pattern) {
    int i;
    int j = -1;
    const int n = strlen(text);
    const int m = strlen(pattern);
    if (n == 0 && m == 0) return 0; /* "","" */
    if (m == 0) return 0;  /* "a","" */
    int *next = (int*)malloc(sizeof(int) * m);

    compute_prefix(pattern, next);

    for (i = 0; i < n; i++) {
        while (j > -1 && pattern[j + 1] != text[i]) j = next[j];
        if (text[i] == pattern[j + 1]) j++;
        if (j == m - 1) { // 全部匹配结束
            free(next);
            return i-j;
        }
    }
    // 匹配失败
    free(next);
    return -1;
}


int main(int argc, char *argv[]) {
    char text[] = "ABC ABCDAB ABCDABCDABDE";
    char pattern[] = "ABCDABD";
    char *ch = text;
    int i = kmp(text, pattern);

    if (i >= 0) printf("matched @: %s\n", ch + i);
    return 0;
}
\end{Codex}

\begin{NoteCode}
#include<iostream>
#include<string>
using namespace std;

void calculte_next(const string &str, int *next){
    const int len = str.length();
    next[0] = 0;
    for(int i = 1, j = 0; i < len; ++i){
        while(j > 0 && str[j] != str[i]) j = next[j];
        if(str[j] == str[i]) j++;
        next[i] = j;
    }
}

int KMP(const string &text, const string &str){
    if(text.length() == 0 || str.length() == 0) return 0;

    const int len_text = text.length();
    const int len_str = str.length();
    int *next = new int[len_text];
    calculte_next(str, next);

    for(int i = 0, j = 0; i < len_text; ++i){
        while (j > 0 && str[j] != text[i]) j = next[j - 1];
        if (str[j] == text[i]) j++;
        if (j == len_str) return i - j + 1;
    }
    return -1;
}

int main(){
    string text = "ABC ABCDAB ABCDABCDABDE";
    string str = "ABCDABD";
    cout << KMP(text, str) << endl;
    return 0;
}
\end{NoteCode}


\subsection{Boyer-Moore算法\Note{暂时先不看2021/4/4}}
详细解释请参考《算法》\footnote{《算法》,Robert Sedgewick,人民邮电出版社,\myurl{http://book.douban.com/subject/10432347/}}第5.3.4节。

推荐网上的几篇比较好的博客,“字符串匹配的Boyer-Moore算法” \myurl{http://www.ruanyifeng.com/blog/2013/05/boyer-moore_string_search_algorithm.html},图文并茂,非常通俗易懂,作者是阮一峰;Boyer-Moore algorithm, \myurl{http://www-igm.univ-mlv.fr/~lecroq/string/node14.html}。

有兴趣的读者还可以看原始论文\footnote{BOYER R.S., MOORE J.S., 1977, A fast string searching algorithm. Communications of the ACM. 20:762-772.}。

Boyer-Moore算法的C语言实现如下。
\begin{Codex}[label=boyer_moore.c]
/**
 * 本代码参考了 http://www-igm.univ-mlv.fr/~lecroq/string/node14.html
 * 精力有限的话,可以只计算坏字符的后移,好后缀的位移是可选的,因此可以删除
 * suffixes(), pre_gs() 函数
 */
#include <stdio.h>
#include <stdlib.h>
#include <string.h>

#define ASIZE 256  /* ASCII字母的种类 */

/*
 * @brief 预处理,计算每个字母最靠右的位置.
 *
 * @param[in] pattern 模式串
 * @param[out] right 每个字母最靠右的位置
 * @return 无
 */
static void pre_right(const char *pattern, int right[]) {
    int i;
    const int m = strlen(pattern);

    for (i = 0; i < ASIZE; ++i) right[i] = -1;
    for (i = 0; i < m; ++i) right[(unsigned char)pattern[i]] = i;
}


static void suffixes(const char *pattern, int suff[]) {
    int f, g, i;
    const int m = strlen(pattern);

    suff[m - 1] = m;
    g = m - 1;
    for (i = m - 2; i >= 0; --i) {
        if (i > g && suff[i + m - 1 - f] < i - g)
            suff[i] = suff[i + m - 1 - f];
        else {
            if (i < g)
                g = i;
            f = i;
            while (g >= 0 && pattern[g] == pattern[g + m - 1 - f])
                --g;
            suff[i] = f - g;
        }
    }
}

/*
 * @brief 预处理,计算好后缀的后移位置.
 *
 * @param[in] pattern 模式串
 * @param[out] gs 好后缀的后移位置
 * @return 无
 */
static void pre_gs(const char pattern[], int gs[]) {
    int i, j;
    const int m = strlen(pattern);
    int *suff = (int*)malloc(sizeof(int) * (m + 1));

    suffixes(pattern, suff);

    for (i = 0; i < m; ++i) gs[i] = m;

    j = 0;
    for (i = m - 1; i >= 0; --i) if (suff[i] == i + 1)
        for (; j < m - 1 - i; ++j) if (gs[j] == m)
            gs[j] = m - 1 - i;
    for (i = 0; i <= m - 2; ++i)
        gs[m - 1 - suff[i]] = m - 1 - i;
    free(suff);
}

/**
 * @brief Boyer-Moore算法.
 *
 * @param[in] text 文本
 * @param[in] pattern 模式串
 * @return 成功则返回第一次匹配的位置,失败则返回-1
 */
int boyer_moore(const char *text, const char *pattern) {
    int i, j;
    int right[ASIZE];  /* bad-character shift */
    const int n = strlen(text);
    const int m = strlen(pattern);
    int *gs = (int*)malloc(sizeof(int) * (m + 1));  /* good-suffix shift */

    /* Preprocessing */
    pre_right(pattern, right);
    pre_gs(pattern, gs);

    /* Searching */
    j = 0;
    while (j <= n - m) {
        for (i = m - 1; i >= 0 && pattern[i] == text[i + j]; --i);

        if (i < 0) { /* 找到一个匹配 */
            /* printf("%d ", j);
            j += bmGs[0]; */
            free(gs);
            return j;
        } else {
            const int max = gs[i] > right[(unsigned char)text[i + j]] -
                    m + 1 + i ? gs[i] : i - right[(unsigned char)text[i + j]];
            j += max;
        }
    }
    free(gs);
    return -1;
}


int main() {
    const char *text="HERE IS A SIMPLE EXAMPLE";
    const char *pattern = "EXAMPLE";
    const int pos = boyer_moore(text, pattern);
    printf("%d\n", pos); /* 17 */
    return 0;
}
\end{Codex}


\subsection{Rabin-Karp算法\Note{暂时先不看2021/4/4}}
详细解释请参考《算法》\footnote{《算法》,Robert Sedgewick,人民邮电出版社,\myurl{http://book.douban.com/subject/10432347/}}第5.3.5节。

Rabin-Karp算法的C语言实现如下。
\begin{Codex}[label=rabin_karp.c]
#include <stdio.h>
#include <string.h>

const int R = 256;  /** ASCII字母表的大小,R进制 */
/** 哈希表的大小,选用一个大素数,这里用16位整数范围内最大的素数 */
const long Q = 0xfff1;

/*
 * @brief 哈希函数.
 *
 * @param[in] key 待计算的字符串
 * @param[int] M 字符串的长度
 * @return 长度为M的子字符串的哈希值
 */
static long hash(const char key[], const int M) {
    int j;
    long h = 0;
    for (j = 0; j < M; ++j) h = (h * R + key[j]) % Q;
    return h;
}

/*
 * @brief 计算新的hash.
 *
 * @param[int] h 该段子字符串所对应的哈希值
 * @param[in] first 长度为M的子串的第一个字符
 * @param[in] next 长度为M的子串的下一个字符
 * @param[int] RM R^(M-1) % Q
 * @return 起始于位置i+1的M个字符的子字符串所对应的哈希值
 */
static long rehash(const long h, const char first, const char next,
                   const long RM) {
    long newh = (h + Q - RM * first % Q) % Q;
    newh = (newh * R + next) % Q;
    return newh;
}

/*
 * @brief Las Vegas version,do real comparison.
 *
 * @param[in] pattern 模式串
 * @param[in] substring 原始文本长度为M的子串
 * @return 两个字符串相同,返回1,否则返回0
 */
static int check(const char *pattern, const char s[]) {
    return strcmp(pattern, s);
}

/*
 * Monte Carlo version, always return true.
 */
static int check(const char *pattern, const char s[]) {
    return 1;
}

/**
 * @brief Rabin-Karp算法.
 *
 * @param[in] text 文本
 * @param[in] n 文本的长度
 * @param[in] pattern 模式串
 * @param[in] m 模式串的长度
 * @return 成功则返回第一次匹配的位置,失败则返回-1
 */
int rabin_karp(const char *text, const char *pattern) {
    int i;
    const int n = strlen(text);
    const int m = strlen(pattern);
    const long pattern_hash = hash(pattern, m);
    long text_hash = hash(text, m);
    int RM = 1;
    for (i = 0; i < m - 1; ++i) RM = (RM * R) % Q;

    for (i = 0; i <= n - m; ++i) {
        if (text_hash == pattern_hash) {
            if (check(pattern, &text[i])) return i;
        }
        text_hash = rehash(text_hash, text[i], text[i + m], RM);
    }
    return -1;
}


int main() {
    const char *text = "HERE IS A SIMPLE EXAMPLE";
    const char *pattern = "EXAMPLE";
    const int pos = rabin_karp(text, pattern);
    printf("%d\n", pos); /* 17 */
    return 0;
}
\end{Codex}

\subsection{总结}
\vspace{1ex}
\begin{center}
\begin{tabular}{llccccc}
\hline
\multirow{2}{*}{\textbf{算法}} & \multirow{2}{*}{\textbf{版本}} & \multicolumn{2}{c}{\textbf{复杂度}} & \textbf{在文本} & \multirow{2}{*}{\textbf{正确性}} & \textbf{辅助}\\
\cline{3-4} & & \textbf{最坏情况} & \textbf{平均情况} & \textbf{中回退} & & \textbf{空间}\\
\hline
\multirow{3}{*}{KMP算法} & 完整的DFA & 2N & 1.1N & 否 & 是 & MR\\
                         & 部分匹配表 & 3N & 1.1N & 否 & 是 & M\\
						 & 完整版本 & 3N & N/M & 是 & 是 & R\\
\hline
Boyer-Moore算法 & 坏字符向后位移 & MN & N/M & 是 & 是 & R\\
\hline
\multirow{2}{*}{Rabin-Karp算法$^*$} & 蒙特卡洛算法 & 7N & 7N & 否 & 是$^*$ & 1\\
                         & 拉斯维加斯算法 & $7N^*$ & 7N & 是 & 是 & 1\\
\hline
\end{tabular}
\end{center}
\small{* 概率保证,需要使用均匀和独立的散列函数}


\section{正则表达式} %%%%%%%%%%%%%%%%%%%%%%%%%%%%%%

% \include{chapStackAndQueue}
% \include{chapTree}
% \include{chapSearching}
% \include{chapSorting}
% \include{chapBruteforce}
% \include{chapBFS}
% \include{chapDFS}
% \include{chapDivideAndConquer}
% \include{chapGreedy}
% \include{chapDynamicProgramming}
% \include{chapGraph}
% \include{chapMath}
% \include{chapBigInt}
% \include{chapFundamental}

\appendix % 开始附录,章用字母编号
\printindex

\end{document}
